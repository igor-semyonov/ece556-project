\documentclass[12pt,english]{article}
\usepackage{authblk}

\usepackage{float}
\usepackage[letterpaper, margin=1in, top=1in, bottom=1in]{geometry}
\usepackage{xcolor}
\usepackage[framemethod=TikZ]{mdframed}
\usepackage{listings, listings-rust}
\usepackage{outlines}
\usepackage{graphicx}
\graphicspath{ {../images/} }
\usepackage{tabularx}
\usepackage{subcaption}
\usepackage{fancyvrb}
\usepackage[english]{babel}
\usepackage{hyperref}

\usepackage{fancyhdr}

\usepackage{amsmath,amsfonts,amssymb,mathrsfs,mathtools}
\usepackage{mathspec}
\DeclarePairedDelimiter\abs{\lvert}{\rvert}%
\DeclarePairedDelimiter\norm{\lVert}{\rVert}%
% Swap the definition of \abs* and \norm*, so that \abs
% and \norm resizes the size of the brackets, and the 
% starred version does not.
\makeatletter
\let\oldabs\abs
\def\abs{\@ifstar{\oldabs}{\oldabs*}}
%
\let\oldnorm\norm
\def\norm{\@ifstar{\oldnorm}{\oldnorm*}}
\makeatother
%%%%%%%%%%%%%%%%%%%

% references for equations
\makeatletter
\def\tagform@#1{\maketag@@@{\bfseries(\ignorespaces#1\unskip\@@italiccorr)}}
\renewcommand{\eqref}[1]{\textup{{\normalfont(\ref{#1}}\normalfont)}}
\makeatother


\usepackage{xparse}

\usepackage{tabto}
\NumTabs{6}

\newcommand{\slide}[2][]{
    \begin{frame}
        \frametitle{
            #1
        }

        #2
    \end{frame}
}

\newcommand{\eq}[2]{
    \begin{equation}\label{#1}
        #2
    \end{equation}
}

\newcommand{\R}{\mathbb{R}}
\newcommand{\Rn}[1][n]{\mathbb{R}^#1}
\NewDocumentCommand{\Rnp}{ O{n} O{p} }{\mathbb{R}^#1_#2}




\title{
    A Spiking Neural Network Implemented in Rust
}
\author{Igor Semyonov}
\author{Kirby Steiner}
\affil{George Mason University}
% \date{}

\begin{document}

\maketitle

\begin{abstract}
    In this work, we explore writing a spiking neural network engine in the Rust programming language.
    We will consider possible methods for training including back propagation and neural generative coding.
    Our implementation is based in part on the work of Eshraghian, et al. \cite{snntorch} and the tutorials of the associated snntorch python package.
    We will profile the code, searching for and comparing optimization strategies.
\end{abstract}

\section{Background}

The project repository is at \href{https://github.com/igor-semyonov/ece556-project}{https://github.com/igor-semyonov/ece556-project}.

\section{Implementation}

Currently, the implementation consists of two components: neuron and layer.
The neuron is a simple leaky integrate and fire (L.I.F.) neuron shown in listing \ref{lst-lif}.
The LIF has two associated functions: new and step.
New is the idiomatic way to write an initialization function.
While it is not necessary, as we can create a LIF struct directly, it allows us to ensure a certain way of creating the struct.
Also, if we make this framework available to others as a library, we could restrict the struct to be a private object, so users would have to use the $new$ function to create instances of LIF.

The $step$ function uses the current state of the LIF, along with input spikes, to return the next membrane potential value and any output spikes, if applicable.
Note, this function does not mutate the LIF in place, it produces the output necessary to update the LIF, but does not do so automatically.
In Rust mutability of variables is strictly controlled and must be managed with care.
I had to change this function to not mutate the LIF in order to allow for downstream use of the LIF struct.

\codefilelines{../src/main.rs}{
    Initial spiking neural network implementation
}{lst-lif}{5}{58}

Layer, shown in listing \ref{lst-layer} is the struct which holds a single layer of a network.
A whole network could also be encoded as a single layer. 
A layer contains one or more neurons, internal weights, memory for all neurons, and spikes in and out for all neurons.
It has three associated functions.
A $new$ function, which like for LIF, is the intended way to instantiate this struct.
The $step$ function performs steps for each neuron and records the spikes for the upcoming step as well as new memory, i.e., membrane potentials.
The $run$ function simply performs the necessary number of steps.
After this the $layer$ struct contains the results of the simulation.

The reason we chose to design it like this is to allow for multiple layers to interact with one another.
Each layer would run its own set of time steps in sequence.
Then it would pass the entire set of spikes to the next layer for downstream processing.

\codefilelines{../src/main.rs}{
    Initial spiking neural network implementation
}{lst-layer}{60}{170}

\section{Demonstration}

In listing \ref{lst-demo} we show the main function from the rust code which creates 3 neurons, an input spike, and a layer.
Then we run the layer and export the results to JSON.
This is made very easy because we derived the Serialize trait for both LIF and layer.
We then use a simple python script shown in listing \ref{lst-vis} to visualize the memory.
The output is shown in figure \ref{fig-mem}

\codefilelines{../src/main.rs}{
    Initial spiking neural network implementation
}{lst-demo}{172}{218}

\codefile{../visualize.py}{
    Visualization script
}{lst-vis}

\begin{figure}[h]
    \centering
    \includegraphics[width=0.8\linewidth]{memory}
    \caption{Memory from 3-neuron-layer with a single input spike propagated through to the other neurons.}
    \label{fig-mem}
\end{figure}

\section{Next Steps}

Now that we have a somewhat functional model, we plan to tear it down and build it following the node-cable structure described by Orbia in \cite{NGC}.
Having built this much in rust already, it should be relatively straight forward to port the NGC architecture to rust, at least the basic components.

I am also considering implementing an evolutionary learning wrapper around the model we build.
This should be relatively doable, and the concurrency libraries in the rust ecosystem are amazing.
For other algorithms I have ported to rust, I have been amazed at how easy it is to perform parallel iteration.
One would think this would be easier in python: a much more high-level language, but with the Rayon crate, I was able to fully utilize all my CPU cores by simply changing one line of code.

%\section*{References}
% \nocite{hybrid-sparcity}
\bibliographystyle{siam}
\bibliography{refs}

%\layout*

\lstlistoflistings

\end{document}
