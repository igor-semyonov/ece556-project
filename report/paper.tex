\documentclass[12pt,english]{article}
\usepackage{authblk}

\usepackage{float}
\usepackage[letterpaper, margin=1in, top=1in, bottom=1in]{geometry}
\usepackage{xcolor}
\usepackage[framemethod=TikZ]{mdframed}
\usepackage{listings, listings-rust}
\usepackage{outlines}
\usepackage{graphicx}
\graphicspath{ {../images/} }
\usepackage{tabularx}
\usepackage{subcaption}
\usepackage{fancyvrb}
\usepackage[english]{babel}
\usepackage{hyperref}

\usepackage{fancyhdr}

\usepackage{amsmath,amsfonts,amssymb,mathrsfs,mathtools}
\usepackage{mathspec}
\DeclarePairedDelimiter\abs{\lvert}{\rvert}%
\DeclarePairedDelimiter\norm{\lVert}{\rVert}%
% Swap the definition of \abs* and \norm*, so that \abs
% and \norm resizes the size of the brackets, and the 
% starred version does not.
\makeatletter
\let\oldabs\abs
\def\abs{\@ifstar{\oldabs}{\oldabs*}}
%
\let\oldnorm\norm
\def\norm{\@ifstar{\oldnorm}{\oldnorm*}}
\makeatother
%%%%%%%%%%%%%%%%%%%

% references for equations
\makeatletter
\def\tagform@#1{\maketag@@@{\bfseries(\ignorespaces#1\unskip\@@italiccorr)}}
\renewcommand{\eqref}[1]{\textup{{\normalfont(\ref{#1}}\normalfont)}}
\makeatother


\usepackage{xparse}

\usepackage{tabto}
\NumTabs{6}

\newcommand{\slide}[2][]{
    \begin{frame}
        \frametitle{
            #1
        }

        #2
    \end{frame}
}

\newcommand{\eq}[2]{
    \begin{equation}\label{#1}
        #2
    \end{equation}
}

\newcommand{\R}{\mathbb{R}}
\newcommand{\Rn}[1][n]{\mathbb{R}^#1}
\NewDocumentCommand{\Rnp}{ O{n} O{p} }{\mathbb{R}^#1_#2}




\title{
    A Spiking Neural Network Implemented in Rust
}
\author{Igor Semyonov}
\author{Kirby Steiner}
\affil{George Mason University}
% \date{}

\begin{document}

% \pagestyle{fancy}
% \fancyhead{}
% \fancyhead[C]{\class}
% \fancyfoot{}
% \fancyfoot[C]{\class}

% the following block, somehow, makes the title appear centered vertically on a cover page.
\null  % Empty line
\nointerlineskip  % No skip for prev line
\vfill
\let\snewpage \newpage
\let\newpage \relax
\maketitle
\thispagestyle{fancy}
\let \newpage \snewpage
\vfill
\break % page break

% include page numbers after title page
\fancyfoot{}
\fancyfoot[C]{
    \thepage \\
    \vspace{1ex}
    \class
}

\begin{abstract}
    In this work, we explore writing a spiking neural network engine in the Rust programming language.
    We will consider possible methods for training including back propagation and neural generative coding.
    Our implementation is based in part on the work of Eshraghian, et al. \cite{snntorch} and the tutorials of the associated snntorch python package.
    We will profile the code, searching for and comparing optimization strategies.
\end{abstract}

\section{Background}

placeholder for background.

\section{Implementation}

Placeholder.

\section{Profiling and Analysis}

placeholder

\section{Results}

Placeholder

\section{Conclusion}

Placeholder

%\section*{References}
% \nocite{detection-algorithms}
% \nocite{hybrid-sparcity}
\bibliographystyle{siam}
\bibliography{refs}

%\layout*

\end{document}
