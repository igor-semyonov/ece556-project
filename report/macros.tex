\usepackage{xparse}

\usepackage{tabto}
\NumTabs{6}

\newcommand{\slide}[2][]{
    \begin{frame}
        \frametitle{
            #1
        }

        #2
    \end{frame}
}

\newcommand{\eq}[2]{
    \begin{equation}\label{#1}
        #2
    \end{equation}
}

\newcommand{\R}{\mathbb{R}}
\newcommand{\Rn}[1][n]{\mathbb{R}^#1}
\NewDocumentCommand{\Rnp}{ O{n} O{p} }{\mathbb{R}^#1_#2}


\renewcommand{\listfigurename}{Figures}

\lstset
{
    language=Rust,
    basicstyle=\footnotesize,
    numbers=left,
    numberstyle=\tiny,
    stepnumber=1,
    showstringspaces=false,
    tabsize=1,
    breaklines=true,
    breakatwhitespace=false,
    xleftmargin=-8pt,
    frame=b,
}

\newcommand{\codefile}[3]{
    \mdframed[roundcorner=5pt, backgroundcolor=blue!30]
    \lstinputlisting[caption=$#1$ {#2}, label=#3]{#1}
    \endmdframed
}
\newcommand{\codefilelines}[5]{
    \mdframed[roundcorner=5pt, backgroundcolor=blue!30]
    \lstinputlisting[style=colouredRust, linerange={#4-#5}, caption=$#1$ Lines #4-#5 {#2}, label=#3]{#1}
    \endmdframed
}
\lstnewenvironment{code}[2]
{
    \mdframed[roundcorner=5pt, backgroundcolor=blue!30]
    \lstset{
        caption={#1},
        label=#2,
    }
}
{
    \endmdframed
}
